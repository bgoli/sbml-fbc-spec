% -*- TeX-master: "main"; fill-column: 72 -*-

\section{Future development}
\label{future}

In this section we highlight some open issues not addressed in this version
of the \FBC specification.

\subsection{Model annotation}
While the \FBCPackage addresses the basic constraint based model structure the
effective annotation of individual model components is critical to its
efficient (re)use. Here two examples of custom annotations are provided as
an example of how this could be achieved.

\subsubsection{Gene association annotation}
An example of an annotation currently encoded in the \Notes element that is
not encodable using the \SBML \Annotation mechanism is the so called ``Gene
Association''.

In order to capture this information we propose the following mechanism
which can be used to represent a logical expression containing ``genes'' in
this case the final (transcribed/translated) protein product of a gene which
forms a sub-component of a protein.
%
\exampleFile{examples/ex_ga_l3.txt}

\subsubsection{A generic annotation mechanism}
An annotation issue previously introduced in the example shown in
\ref{examples} where a \SBML Level 2 \Reaction is encoded using the
\FBCPackage.

\paragraph{\SBML Level 2 \Reaction}
%
\exampleFile{examples/ex_reaction_bigg.txt}

\paragraph{\SBML Level 3 \Reaction}
Note how that in order to maintain all the annotations encoded in the \SBML
Level 2 \Reaction \token{notes} a custom annotation is introduced i.e.~the
\textsf{KeyValueData} class\footnote{More information about this annotation
is available at \url{http://pysces.sourceforge.net/KeyValueData}}.

In general this example highlights the need for a community supported
annotation mechanism for genome scale, constraint based models.
%
\exampleFile{examples/ex_reaction_l3.txt}
